\documentclass[b5paper,10pt]{article}

\usepackage{url} %this package should fix any errors with URLs in refs.
% Images
\usepackage{graphicx}
% Citations
\usepackage[numbers,compress]{natbib}

% Margins
\usepackage[b5paper,left=20mm,right=20mm,top=20mm,bottom=20mm]{geometry}
\usepackage{t1enc}
\usepackage[utf8]{inputenc}
\usepackage[magyar]{babel}
% Affiliation
\usepackage{authblk}

%\linenumbers

%\draftfalse

%\journalname{Space Weather}

% Caption skip
\setlength{\belowcaptionskip}{-10pt}
\setlength{\abovecaptionskip}{0pt plus 3pt minus 2pt}

\begin{document}

\title{A földi magnetoszféra egy év hosszú magnetohidrodinamikai szimulációjának összehasonlítása Cluster mérésekkel}

\author[,1,2]{FACSKÓ~Gábor\thanks{Levelező szerző: Facsk{\'o} G{\'a}bor (facsko.gabor@wigner.hu, gabor.i.facsko@gmail.com)}}
\author[3]{SIBECK~David}
\author[4]{HONKONEN~Ilja}
\author[5]{BÓR~József}
\author[6]{FARINAS~PEREZ~German}
\author[1]{TIMÁR~Anikó}
\author[7,8]{SHPRITS~Yuri}
\author[4]{DEGENER~Laura}
\author[9,10]{TANSKANEN~Eija}
\author[10]{PEITSO~Pyry}
\author[11]{ANEKALLU~Chandrasekhar~Reddy}
\author[5,12]{SZALAI~Sándor}
\author[5]{KIS~Árpád}
\author[5]{WESZTERGOM~Viktor}
\author[1,13]{MADÁR~Ákos}
\author[1,13]{BIRÓ~Nikolett}
\author[1,13]{KOBÁN~Gergely}
\author[1]{ILLYÉS~András}
\author[1,14]{LKHAGVADORJ~Munkhjargal}

\affil[1]{\footnotesize Wigner Fizikai Kutatóközpont, Űrfizikai és Űrtechnikai Osztály, Budapest}
\affil[2]{Milton Friedman Egyetem, Informatika Tanszék, Budapest}
\affil[3]{NASA Goddard Space Flight Center, Greenbelt, MD, USA}
\affil[4]{Finnish Meteorological Institute, Helsinki, Finland}
\affil[5]{Földfizikai és Űrtudományi Kutatóintézet (ELKH FI), Sopron}
\affil[6]{University of Miami, Electrical and Computer Engineering Department, Miami, Florida, USA}
\affil[7]{Helmholtz Centre Potsdam, GFZ German Research Centre for Geosciences, Potsdam, Germany}
\affil[8]{Institute for Physics and Astronomy, University of Potsdam, Potsdam, Germany}
\affil[9]{Sodankyl{\"a} Geophysical Observatory, University of Oulu, Sodankyl{\"a}, Finland}
\affil[10]{Department of Electronics and Nanoengineering, Aalto University, Espoo, Finland}
\affil[11]{UCL Department of Space \& Climate Physics, Mullard Space Science Laboratory, Dorking, UK}
\affil[12]{Miskolci Egyetem, Geofizika Tanszék, Miskolc}
\affil[13]{Eötvös Loránd Tudományegyetem, Fizika Doktori Iskola, Budapest}
\affil[14]{Eötvös Loránd Tudományegyetem, Természettudományi Kar, Budapest}

\date{ }

\maketitle

\vspace{-1cm}

\begin{abstract}
\textit{Korábban a 2002. január 29. és 2003. február 2. közötti időszak napszéladatait felhasználva egy év hosszúságú időszakra globális magnetohidrodinamikai szimulációt készítettünk a GUMICS$-$4 kóddal. Ebben a tanulmányban összehasonlítjuk a modellt a földi magnetoszféra és a napszél kölcsönhatásának a tekintetében a Cluster SC3 méréseivel. A szimulációból kapott megfelelő adatokat hasonlítjuk össze a magnetométer által észlelt bolygóközi mágneses tér észak-déli komponensével, az ion plazma műszerrel mért Nap-Föld egyenessel párhuzamos napszélsebességgel és az ion plazma sűrűséggel, továbbá az elektromos teret mérő műszer által szolgáltatott szonda potenciálból számolt elektron sűrűséggel. Meghatározzuk a lökéshullám, a magnetopauza és a semleges lepel koordinátáit az űrszonda méréseiből és a szimulációkból, majd összehasonlítjuk a helyzetüket. Olyan időszakaszokat választunk a napszélben, a mágneses burokban és a magnetoszférában, ahol a fenti műszerek jó minőségű adatokat biztosítanak és ahol modell szerint is abban a régióban tartózkodott a szonda, ahol a valóságban. A GUMICS$-$4 kód eredményei jól egyeznek a mérésekkel a napszélben, de a pontosság csökken a mágneses burokban. A magnetoszférában a szimuláció eredményei nem valósághűek. A lökéshullám helyzetét jól adja vissza a modell, azonban a magnetopauza helyzete kevésbé pontos. A semleges lepel helyzete annak köszönhetően valósághű, hogy a bolygóközi mágneses tér $B_{y}$ komponense kicsi volt ebben az időszakban.}
\end{abstract}

\section*{Bevezetés}
\label{sec:bevezetes}

Az egyik legköltséghatékonyabb módszer a napszél és a bolygók magnetoszférájának a tanulmányozásához és a Föld-közeli űr tulajdonságainak előjelzéséhez a komplex rendszer magnetohidrodinamikai (MHD) modellezése. Korábban több ilyen kódot fejlesztettek, amelyeket a Föld-közeli űr viszonyainak előrejelzésére használtak \cite{facsko21:_compar_gumic_terres_magnet_clust_measur}. A szimulációk eredményeinek az összehasonlítása űrszondás, vagy földi mérésekkel létfontosságú a fejlesztett kódok tulajdonságainak megértése szempontjából. Korábban egy év hosszú MHD szimulációt futtattunk le a GUMICS$-$4 kóddal 2002.~január~29. és 2003.~február~2. közötti OMNI napszél adatokat használva bemenetként \citep{facsko16:_one_earth}. Ezt hasonlítjuk össze az Európai Űrügynökség (European Space Agency, ESA) Cluster-II missziójának méréseivel, amelyet 2000-ben bocsátottak fel a Föld közeli űr tanulmányozására \cite{escoubet01:_introd_clust}. A mágneses mezőt a FluxGate Magnetometer (FGM) \cite{balogh01:_clust_magnet_field_inves}, a plazma ion adatokat a Cluster Ion Spectrometry (CIS) műszer Hot Ion Analyser (HIA) alműszere méri \citep{reme01:_first_earth_clust_cis}. A CIS HIA műszert a Cluster fedélzetén található Waves of HIgh frequency Sounder for Probing the Electron density by Relaxation (WHISPER) elektrosztatikus hullámműszerrel kalibrálják \cite{trotignon10:_whisp_relax_sound_clust_activ_archiv,blagau14:_in_hot_ion_analy_clust}. Az E\-lec\-tric Field and Wave Experiment (EFW) műszer \cite{gustafsson01:_first_clust_efw} által szolgáltatott űrhajó potenciál segítségével határozzuk meg a elektronok sűrűségét az $n_{EFW}=200(V_{sc})^{-1.85}$ empirikus sűrűségformula segítségével, ahol $n_{EFW}$ a számolt sűrűsség és $V_{sc}$ a Cluster EFW űrhajó potenciál \cite{trotignon10:_whisp_relax_sound_clust_activ_archiv,trotignon11:_calib_repor_whisp_measur_clust}. Ez a cikk a Cluster SC3 méréseit közvetlenül veti össze az egy év hosszú GUMICS$-$4 szimulációkkal a napszélben, a mágneses burokban és a magnetoszférában a referencia szonda pályája mentén \cite{facsko16:_one_earth}. A vizsgált paraméterek a mágneses tér északi/déli komponense GSE koordinátákban ($B_{z}$), a napszél sebesség GSE Nap-Föld egyenessel párhuzamos komponense ($V_x$) és a napszél ion plazma ($n_{CIS}$) és számolt elektron sűrűsége ($n_{EFW}$). Összehasonlítottuk továbbá a jósolt és megfigyelt helyzetét a lökéshullámnak, a magnetopauzának és a semleges lepelnek. Ezeket a paramétereket azért választottuk, mert $B_z$ kontrollálja a napszél--magnetoszféra kölcsönhatást, a $V_x$ a napszél fő sebesség komponense és az $n_{CIS}$ ion plazma momentumot a legegyszerűbb kiszámolni, továbbá több műszer is képes mérni ezt a mennyiséget.


\section*{Mérések és szimulációk összehasonlítása}
\label{sec:osszehasonlitas}

\begin{figure}[t]
\centering
\includegraphics[width=0.9\textwidth,angle=0]{mufcikk-f01}
\caption{A Cluster SC3 mérések és a GUMICS$-$4 szimulációk összehasonlító ábrái az összes vizsgált intervallumra a napszélben. A szaggatott vonal az y=x függvény. (a) A mágneses tér Z komponense GSE koordináta-rendszerben. (b) A napszél sebesség X komponense GSE koordináta-rendszerben. (c) A CIS HIA műszer által mért ion plazma (szürke) és az űrhajó potenciálból számolt elektron sűrűség (fekete).}
\label{fig:swscatplot}
\end{figure}

A GUMICS$-$4 szimulációból a Cluster SC3 pályája mentén elmentett paramétereket és a szonda mágneses, napszél sebesség és sűrűség méréseit hasonlítjuk össze a napszélben, a mágneses burokban és magnetoszférában keresztkorrelációs számításokkal. A szimulált adatok időfelbontása öt perc \cite{facsko16:_one_earth}. Az analízishez a különböző időfelbontású adatsorok összehasonlíthatósága érdekében egy perces felbontásra interpoláljuk az eredményeket. Az adathiányokat lineáris interpolációval töltöttük fel, illetve extrapolációval, ha az adathiány a kiválasztott intervallum elején, vagy végén található. A négy másodperces felbontású mágneses tér méréseket öt perces felbontásúra átlagoltuk, majd egy perces felbontásúra interpoláltuk, hogy el tudjuk végezni a korrelációs számításokat. A kiválasztott intervallumokban a paraméterek nem változnak sokat, továbbá semmilyen határfelületet nem keresztez sem a Cluster SC3, sem a virtuális szonda a szimulációban. 

Az OMNI adatbázis bolygóközi mágneses mező, sebesség, sűrűség és hőmérséklet adatait adtuk meg a szimuláció bementi adatainak \cite{facsko16:_one_earth}. Összesen csak 17 megfelelő intervallumot használunk fel a tanulmányban \cite{facsko21:_compar_gumic_terres_magnet_clust_measur}. A $B_{z}$, $V_{x}$ és $n_{EFW}$ nagyon jól egyezik a napszélben a szimulációkkal (\ref{fig:swscatplot}a,~\ref{fig:swscatplot}b, \ref{fig:swscatplot}c~ábra,~fekete). Az $n_{CIS}$ nem egyezik olyan jól (\ref{fig:swscatplot}c.~ábra,~szürke), ami várható volt, mivel a CIS műszernek sok üzemmódja van a plazma paramétereinek a mérésére és folyamatos kalibrációra van szüksége, ezért a mindenki által hozzáférhető adatai sokszor pontatlanok. A napszélben $V_{x}$, $n_{CIS}$ és $n_{EFW}$ paraméterek korrelációja a megegyező GUMICS szimulációs paraméterekkel nagyobb 0.9-nél, azaz kitűnő \cite{facsko21:_compar_gumic_terres_magnet_clust_measur}. A $B_{z}$ korrelációja nagyobb 0.8-nál, vagyis nagyon jó \cite{facsko21:_compar_gumic_terres_magnet_clust_measur}. Az időeltolódás kb.~öt perc a mágneses térre és a CIS adatokra \cite{facsko21:_compar_gumic_terres_magnet_clust_measur}.  A GUMICS$-$4 szimulációs terének upstream határa $32\,R_{F\ddot{o}ld}$-nál van \cite{janhunen12:_gumic_mhd}, a Föld lökéshullámának az orra pedig kb.~$20\,R_{F\ddot{o}ld}$-nél. Ha napszél sebessége 400\,km/s, akkor ez a távolság kevesebb, mint öt perc késést okoz, amelynek nem lenne szabad látszódnia a keresztkorreláció eredményében. Az EFW származtatott adataira az időeltolódás kevésbé kifejezett \cite{facsko21:_compar_gumic_terres_magnet_clust_measur}.

\begin{figure}[!t]
\centering
\includegraphics[width=0.9\textwidth,angle=0]{mufcikk-f02}
\caption{A Cluster SC3 mérések és a GUMICS$-$4 szimulációk összehasonlító ábrái az összes vizsgált intervallumra a mágneses burokban. A szaggatott vonal az y=x függvény. (a) A mágneses tér Z komponense GSE koordináta-rendszerben. (b) A napszél sebesség X komponense GSE koordináta-rendszerben. (c) A CIS HIA műszer által mért (szürke) és az űrhajó potenciálból számolt sűrűség (fekete).}
\label{fig:mshscatplot}
\end{figure}

A Cluster SC3 minden keringés során elmerül a mágneses burokban 2002. decembere és a 2003. májusa között, ennélfogva 74 intervallumot sikerült kiválasztani \cite{facsko21:_compar_gumic_terres_magnet_clust_measur}. A mágneses mező, a napszélsebesség és sűrűségek összehasonlító ábráin a paraméterek jól egyeznek, de nagyobb variációval, mint a napszélben (\ref{fig:mshscatplot}a,~\ref{fig:mshscatplot}b,~\ref{fig:mshscatplot}c~ábra). A szimuláció és az észlelések jól egyeznek a mágneses mező esetén \cite{facsko21:_compar_gumic_terres_magnet_clust_measur} és nagyon jól egyeznek a ion plazma momentumok és a származtatott sűrűség esetén \cite{facsko21:_compar_gumic_terres_magnet_clust_measur}.  Míg a paraméterek korreláltak, a szórás nagyobb. Ennek oka az lehet, hogy a mágneses burok erősen turbulens, amely megmagyarázza, hogy a $B_{z}$ mágneses mező nagyobb változékonyságát a \ref{fig:mshscatplot}a~ábrán. A napszélbeli $V_{x}$, $n_{CIS}$ és $n_{EFW}$ jobban egyezik a mágneses mező komponensénél (\ref{fig:mshscatplot}b,~\ref{fig:mshscatplot}c~ábra), azonban a korreláció még mindig jó \cite{facsko21:_compar_gumic_terres_magnet_clust_measur}. A többi paraméter korrelációja nagyobb ($0.9$), de az időeltolódás rosszabbnak tűnik \cite{facsko21:_compar_gumic_terres_magnet_clust_measur}. Ennek oka az, hogy az idősorok igen simák a mágneses burokban, így nincs elég pont, hogy egy éles csúcsot alkosson a korrelációs függvény. A maximum és a minimum közötti különbség kicsi, a számítás hibájával összehasonlítva. A maximum, az időeltolódás bárhol lehet és a korreláció~vs.~időletolódás függvény gyakran nem szimmetrikus, továbbá nem csak egy lokális maximuma van. Ennélfogva a korreláció számítás nagyobb időeltolódásokat eredményez az ionplazma paraméterekben, mint a $n_{EFW}$-ben \cite{facsko21:_compar_gumic_terres_magnet_clust_measur}. A GUMICS$-$4 kevésbé pontos a mágneses burokban, mint a napszélben, továbbá a modellezett mágneses mező jobban egyezik a mérésekkel, mint a modellezett plazma paraméterek. A számított EFW sűrűség ($n_{EFW}$) jobban egyezik a szimulációkkal a CIS HIA sűrűségnél ($n_{CIS}$) \cite{facsko21:_compar_gumic_terres_magnet_clust_measur}.

132 intervallumot sikerült találni, amikor a Cluster SC3 és a virtuális szonda is magnetoszférában tartózkodik \cite{facsko21:_compar_gumic_terres_magnet_clust_measur}. A magnetoszférában egyetlen kiválasztott paraméter sem egyezik a mérésekben és a szimulációkban. Pl. a napszél sebessége nem csökken zérusra a magnetoszférában, ehelyett a napszél behatol a geomágneses csóvába. A napszélben a CIS HIA által mért és a EFW űrszonda potenciálból származtatott sűrűségek növekednek, ahogy a Földhöz közelítünk (plazmaszféra), míg a GUMICS$-$4 sűrűség egyszerűen csak alacsony. A GUMICS$-$4 egy dipólus modellt használ a Föld mágneses mezejének a leírására \cite{janhunen12:_gumic_mhd}. A modell mágneses tér levonása után a Cluster SC3 mágneses tér mérései és a szimulációk továbbra is igen kicsi korrelációt mutatnak az időeltolódás is irreális. Ennélfogva a dipólus modell elégtelen leírása a belső magnetoszféra mágneses mezejének. Ezen felül sem a plazma momentumok, sem a $n_{EFW}$ nem illeszkedik. Az egy folyadékos ideális MHD nem írja le jól a belső magnetoszférát, ennélfogva a szimulációbeli $V_{x}$ és az $n$ nem egyezik jól a Cluster SC3 által mért adatokkal \cite{facsko21:_compar_gumic_terres_magnet_clust_measur}.

\begin{figure}[t]
\centering
\includegraphics[width=0.9\textwidth,angle=0]{mufcikk-f03} 
\caption{A mágneses tér $B_{z}$ komponense a GUMICS$-$4 szimulációk eredményeiben (folytonos vonal) és a Cluster SC3 méréseiben (szaggatott vonal) 2002. szeptember 28. 03:00-tól 07:00-ig (UT) GSE koordinátákban a geomágneses csóvában. 05:15-től 05:30-ig a két függőleges szaggatott vonal között a Cluster SC3 és a virtuális szonda többször keresztezi a semleges leplet.}
\label{fig:nsplot}
\end{figure}

77 intervallumot sikerült kiválasztani, amikor a Cluster SC3 a földi lökéshullámon egyszer, vagy többször áthalad \cite{facsko21:_compar_gumic_terres_magnet_clust_measur}. A virtuális szonda lökéshullám átmenetei lassabbak és simábbak, ezen felül a GUMICS$-$4 nem mutatja a többszörös keresztezéseket. A kód lassan reagál a hirtelen változásokra a napszélben. A mágneses mező jobban illeszkedik a mérésekhez, mint a plazma momentumok. A CIS HIA és az EFW sűrűség ugrások eltolódtak a szimulációkhoz képes. A sűrűség és sebesség a szimulációkban kevésbé pontos, mint a mágneses mező \cite{facsko21:_compar_gumic_terres_magnet_clust_measur}. 54 intervallumot sikerült kiválasztani magnetopauza áthaladások körül \cite{facsko21:_compar_gumic_terres_magnet_clust_measur}. A magnetopauza helyzete jól meghatározott a Cluster SC3 mérésekben, de nagyon nehéz azonosítani a szimulációkban. Az esetek többségében (92\,\%) nem látható a magnetopauza a szimulációkban, továbbá egyáltalán nem látható $V_{x}$-ben és n-ben. Ez a megfigyelés független a bolygóközi mágneses mező irányától. Sokszor, amikor az átmenet a mérésekben és a szimulációkban is látható, az esemény térben és időben eltolódva látható a szimulációkban és a mérésekben. A magnetopauza helyzetét a modell kevésbé pontosan határozza meg, mint a lökéshullámét. Ez az eltérés a pozíciókban jól egyezik a korábbi eredményekkel \cite{gordeev13:_verif_gumic_mhd,facsko16:_one_earth}, amikor szintetikus GUMICS futtatásokat hasonlítottak össze empirikus magnetopauza formulákkal \cite{gordeev13:_verif_gumic_mhd} és egy év hosszú OMNI adatokat használó GUMICS szimuláció \cite{facsko16:_one_earth} megerősítette ezt a következtetést \cite{facsko21:_compar_gumic_terres_magnet_clust_measur}. Kilenc semleges lepel keresztezést sikerült találni a Cluster SC3 adataiban \cite{facsko21:_compar_gumic_terres_magnet_clust_measur}. Öt esetben a GUMICS$-$4 szintén $B_{z}$ előjelváltozást mutat a simított görbékben, azaz a semleges lepel látható mind a szimulációban, mind a mérésekben (\ref{fig:nsplot}.~ábra; folytonos vonal). Ez kiemelkedő eredmény, mert a GUMICS$-$4 szimulációban a geomágneses csóva jelentősen rövidebb, mint a valóságban \cite{gordeev13:_verif_gumic_mhd,facsko16:_one_earth}; továbbá a napszél általában az MHD szimulációkban behatol a csóvába \cite{kallio15:_proper}. Azonban ebben az esetben a bolygóközi térnek nincsen nagy $B_{y}$ komponense és korábbi munkákból tudjuk, hogy ebben az esetben a GUMICS geomágneses csóvája (vagy éjszakai magnetoszférája) normális hosszúságú \cite{facsko16:_one_earth, facsko21:_compar_gumic_terres_magnet_clust_measur}. A GUMICS kód képes visszaadni a lökéshullám átmeneteket. A magnetopauza és a semleges lepel esetén a helyzet sokkal komplexebb.


\section*{Összegzés}
\label{sec:osszegzes}

Korábban készített egy év hosszú GUMICS$-$4 szimulációk eredményét hasonlítjuk össze a Cluster SC3 szonda mágneses mező, napszél sebesség és sűrűség méréseivel a műhold pályája mentén. Olyan intervallumokat választunk ki, amikor a műhold és a virtuális szonda egyszerre tartózkodnak a napszélben, a mágneses burokban és a magnetoszférában, majd keresztkorrelációt számolunk ezek között az idősorok között. Lökéshullám, magnetopauza és a semleges lepel áthaladásokat vizsgálunk és összehasonlítjuk az észlelhetőségüket, továbbá az egymáshoz viszonyított helyzetüket. A vizsgálat a következő eredménnyel zárult:
\begin{enumerate}
\setlength{\parskip}{-3pt}
\item A napszélben a $B_{z}$ korrelációja nagyobb, mint 0.8. A $V_{x}$, az $n_{EFW}$ és az $n_{CIS}$ korrelációja pedig nagyobb 0.9-nál. A $B_{z}$, a $V_{x}$, és a $n_{EFW}$ nagyon jól egyezik, továbbá a $n_{CIS}$ egyezése is jó.
\item A mágneses burokban a $B_{z}$ korrelációs együtthatója nagyobb, mint 0.6. A $V_{x}$, az $n_{EFW}$ és az $n_{CIS}$ korrelációja pedig nagyobb 0.9-nál. A mágneses mező komponens, az ion plazma momentumok és a számított empirikus sűrűség egyezése egy kicsit gyengébb, mint a napszélben. A mágneses burokban a $V_{x}$, az $n_{EFW}$ és az $n_{CIS}$ jobban illeszkedik, mint a $B_{z}$ komponens. Ez az egyezés még mindig nagyon jó. A nagyobb eltérés oka a lelassult, felhevült és turbulens napszél a lökéshullám után, amely látszik a Cluster SC3 méréseiben, de a szimulációkban nem.   
\item Sem a nappali, sem az éjszakai oldali magnetoszférában nem képes a GUMICS$-$4 realisztikus eredményeket produkálni, azaz a szimulációk kimenete és az űrszondás mérések még csak nem is hasonlítanak egymásra ebben a régióban. Ennek az az oka, hogy a GUMICS$-$4-ben nincsen belső magnetoszféra modell. 
\item Ebben a tanulmányban a lökéshullám és a semleges lepel pozíciója jól egyezik a szimulációkban és a Cluster SC3 mágneses mező, ion plazma momentum és származtatott elektron sűrűség méréseiben. A magnetopauza helyzete nem illeszkedik ennyire jól. 
\end{enumerate}
A GUMICS$-$4-nek tudományos és stratégiai jelentősége van az európai űridőjárás előrejelző és tudományos közössége számára. Ez a Finn Meteorológiai Intézetben fejlesztett kód a legfejlettebb, legjobban tesztelt és a legszélesebb körben használt eszköz a Föld kozmikus környezetének modellezésére Európában. A szimulációk pontosságának növelésére a jelenlegi kódhoz az ionoszférához és a magnetohidrodinamikai tartományhoz egyaránt csatolt belső magnetoszféra modellt kellene hozzáadni. 

\paragraph{Köszönetnyilvánítás} Az OMNI adatok a GSFC/SPDF OMNIWeb NASA adatbázisból (http://omniweb.gsfc.nasa.gov) származnak. A szerzők köszönik az FGM~(PI:~C.~Carr), a CIS~(PI:~I.~Dandouras), a WHISPER~(PI:~J-L.~Rauch) és az EFW~(PI: M.~Andre) műszercsapatoknak, továbbá a Cluster Science Archive adatbázisnak az FGM mágneses mező, a CIS~HIA ion plazma és a WHISPER elektron sűrűség méréseket. Az adatok elemzését részben a  QSAS tudományos elemző rendszerrel végeztük, amelyet a United Kingdom Cluster Science Centre (Imperial College London and Queen Mary, University of London) fejlesztett a Science and Technology Facilities Council (STFC) támogatásával. TANSKANEN Eija nagyon köszöni a Finn Akadémia anyagi támogatását a ReSoLVE Centre of Excellence (No.~272157) projekthez. A szerzők rendkívül hálásak JANHUNEN Pekkának a GUMICS$-$4 kód kifejlesztésért; továbbá, PALMROTH Minnának, OPITZ Andreának, KALM{\'A}R J{\'a}nosnak és V{\"O}R{\"O}S Zolt{\'a}nnak a hasznos tanácsokért. FACSKÓ Gábor munkáját a Van Allen Probes misszió, az Amerikai Geofizikai Unió, és az ELKH Földfizikai és Űrtudományi Kutatóintézet, Sopron támogatta. LKHAGVADORJ Munkhjargal munkáját a Stipendium Hungaricum ösztöndíj támogatta. Ezt a tanulmányt részben a Nemzeti Kutatási, Fejlesztési és Innovációs Hivatal (NKFIH) FK128548 támogatása finanszírozta.

\bibliography{mufcikk-tx}
\bibliographystyle{aa}

\end{document}